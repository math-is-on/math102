% Created 2024-03-17 Sun 21:09
% Intended LaTeX compiler: pdflatex
\documentclass[presentation]{beamer}
\usepackage[utf8]{inputenc}
\usepackage[T1]{fontenc}
\usepackage{graphicx}
\usepackage{longtable}
\usepackage{wrapfig}
\usepackage{rotating}
\usepackage[normalem]{ulem}
\usepackage{amsmath}
\usepackage{amssymb}
\usepackage{capt-of}
\usepackage{hyperref}
\mode<beamer>{\usetheme{Madrid}}
\definecolor{SUred}{rgb}{0.59375, 0, 0.17969} % SU red (primary)
\definecolor{SUblue}{rgb}{0, 0.17578, 0.38281} % SU blue (secondary)
\setbeamercolor{palette primary}{bg=SUred,fg=white}
\setbeamercolor{palette secondary}{bg=SUblue,fg=white}
\setbeamercolor{palette tertiary}{bg=SUblue,fg=white}
\setbeamercolor{palette quaternary}{bg=SUblue,fg=white}
\setbeamercolor{structure}{fg=SUblue} % itemize, enumerate, etc
\setbeamercolor{section in toc}{fg=SUblue} % TOC sections
% Override palette coloring with secondary
\setbeamercolor{subsection in head/foot}{bg=SUblue,fg=white}
\setbeamercolor{date in head/foot}{bg=SUblue,fg=white}
\institute[SU]{Shenandoah University}
\titlegraphic{\includegraphics[width=0.5\textwidth]{\string~/Documents/suLogo/suLogo.pdf}}
\newcommand{\R}{\mathbb{R}}
\usetheme{default}
\author{Chase Mathison\thanks{cmathiso@su.edu}}
\date{18 March 2024}
\title{Trigonometric Identities}
\hypersetup{
 pdfauthor={Chase Mathison},
 pdftitle={Trigonometric Identities},
 pdfkeywords={},
 pdfsubject={},
 pdfcreator={Emacs 29.1 (Org mode 9.6.7)}, 
 pdflang={English}}
\begin{document}

\maketitle

\section{Announcements}
\label{sec:org9267c3d}
\begin{frame}[label={sec:org917eb12}]{Announcements}
\begin{enumerate}
\item Homework in M.O.M.
\item Exam 2 on Friday.
\end{enumerate}
\end{frame}

\section{Lecture}
\label{sec:org6bfa320}
\begin{frame}[label={sec:org6c198f2}]{Verifying Trigonometric Identities}
Today we're going to talk about how to verify trig identities.  Let's start with an example:

We want to \emph{verify} that
\[
\sin(x)\sec(x) = \tan(x)\]

Let's see how we do that:
\vspace{10in}
\end{frame}

\begin{frame}[label={sec:orgf4131d9}]{Verifying Trigonometric Identities}
\end{frame}

\begin{frame}[label={sec:org65a4282}]{Basic Principals of Verifying Trig Identities}
Here are 4 basic principals for verifying trig identities:
\begin{enumerate}
\item Always work on only one side of the equation.  Try to work on the most \uline{\hspace*{1in}} side. (It's easier to \uline{\hspace*{1in}} than to build)
\item Look for opportunities to \uline{\hspace*{1in}}, \uline{\hspace*{1in}} or add \uline{\hspace*{1in}}.
\item Note which functions are in the final expression, and look for opportunities to make appropriate \uline{\hspace*{1in}}.
\item If all else fails, try changing everything to \uline{\hspace*{1in}} and \uline{\hspace*{1in}}.
\end{enumerate}
\end{frame}

\begin{frame}[label={sec:orgb8f20df}]{Fundamental Identities}
The following are fundamental identities that you can take for granted when trying to verify other trig identities:

\begin{block}{Pythagorean Identities}
\[\sin^2 \left( x \right) + \cos^2 \left( x \right) = \]

\[
\tan^2(x) + 1 = \]

\[
1 + \cot^2(x) = \]
\end{block}


\begin{block}{Even/Odd identities}
\begin{enumerate}
\item \(\sin(-x) =\)
\item \(\cos(-x) =\)
\item \(\tan(-x) =\)
\item \(\csc(-x) =\)
\item \(\sec(-x) =\)
\item \(\cot(-x) =\)
\end{enumerate}
\end{block}
\end{frame}

\begin{frame}[label={sec:orgde202a8}]{Fundamental Identities}
\begin{block}{Recipricol Identities}
\begin{enumerate}
\item \(\csc(x) =\)
\item \(\sec(x) =\)
\item \(\cot(x) =\)
\end{enumerate}
\end{block}

\begin{block}{Quotient Identities}
\begin{enumerate}
\item \(\tan(x) =\)
\item \(\cot(x) =\)
\end{enumerate}
\end{block}
\end{frame}


\begin{frame}[label={sec:orgcdfbd31}]{Example}
Let's verify another trig identity:

\[
\frac{\sin(x)}{\tan(x)} = \cos(x)
\]

\vspace{10in}
\end{frame}
\begin{frame}[label={sec:orgea8b093}]{Example}
\end{frame}

\begin{frame}[label={sec:org98ae5cc}]{Example}
Verify that
\[
\csc(x)\cos(x)\tan(x) = 1\]
\vspace{10in}
\end{frame}
\begin{frame}[label={sec:orgabd23a7}]{Example}
\end{frame}

\begin{frame}[label={sec:orga2a3f09}]{Example}
Verify
\[
\left( 1+\sin(x) \right) \left( 1 + \sin(-x) \right) = \cos^2(x)\]
\vspace{10in}
\end{frame}

\begin{frame}[label={sec:orgff46c67}]{Example}
\end{frame}

\begin{frame}[label={sec:org7c049db}]{Example}
Verify
\[
\sin(x) \cos(x) + \sin^2(x) \tan(x) = \tan(x)\]

\vspace{10in}
\end{frame}

\begin{frame}[label={sec:orgc30b74a}]{Example}
\end{frame}

\begin{frame}[label={sec:org005451a}]{Example}
Verify \[ \frac{\sin^2(x) - \cos^2(x)}{\sin(x) - \cos(x)} = \sin(x) +
\cos(x).\]

\vspace{10in}
\end{frame}

\begin{frame}[label={sec:org64303a6}]{Example}
\end{frame}

\begin{frame}[label={sec:org4b49f91}]{Example}
One more!

Verify
\[
\frac{\cos(x)}{1+\sin(x)} = \frac{1 - \sin(x)}{\cos(x)}\]

\vspace{10in}
\end{frame}

\begin{frame}[label={sec:org9b249b6}]{Example}
\end{frame}

\begin{frame}[label={sec:orgac792e1}]{Example}
How can we see a claimed identity is false?  For example: Is it true that
\[
\sin(2x) = 2\sin(x)?\]
\vspace{10in}
\end{frame}
\end{document}