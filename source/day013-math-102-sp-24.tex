% Created 2024-02-21 Wed 07:59
% Intended LaTeX compiler: pdflatex
\documentclass[presentation]{beamer}
\usepackage[utf8]{inputenc}
\usepackage[T1]{fontenc}
\usepackage{graphicx}
\usepackage{longtable}
\usepackage{wrapfig}
\usepackage{rotating}
\usepackage[normalem]{ulem}
\usepackage{amsmath}
\usepackage{amssymb}
\usepackage{capt-of}
\usepackage{hyperref}
\mode<beamer>{\usetheme{Madrid}}
\definecolor{SUred}{rgb}{0.59375, 0, 0.17969} % SU red (primary)
\definecolor{SUblue}{rgb}{0, 0.17578, 0.38281} % SU blue (secondary)
\setbeamercolor{palette primary}{bg=SUred,fg=white}
\setbeamercolor{palette secondary}{bg=SUblue,fg=white}
\setbeamercolor{palette tertiary}{bg=SUblue,fg=white}
\setbeamercolor{palette quaternary}{bg=SUblue,fg=white}
\setbeamercolor{structure}{fg=SUblue} % itemize, enumerate, etc
\setbeamercolor{section in toc}{fg=SUblue} % TOC sections
% Override palette coloring with secondary
\setbeamercolor{subsection in head/foot}{bg=SUblue,fg=white}
\setbeamercolor{date in head/foot}{bg=SUblue,fg=white}
\institute[SU]{Shenandoah University}
\titlegraphic{\includegraphics[width=0.5\textwidth]{\string~/Documents/suLogo/suLogo.pdf}}
\newcommand{\R}{\mathbb{R}}
\usetheme{default}
\author{Chase Mathison\thanks{cmathiso@su.edu}}
\date{21 February 2024}
\title{Modeling with Sinusoidal Functions}
\hypersetup{
 pdfauthor={Chase Mathison},
 pdftitle={Modeling with Sinusoidal Functions},
 pdfkeywords={},
 pdfsubject={},
 pdfcreator={Emacs 29.1 (Org mode 9.6.7)}, 
 pdflang={English}}
\begin{document}

\maketitle

\section{Announcements}
\label{sec:org832d302}
\begin{frame}[label={sec:org3cb3df4}]{Announcements}
\begin{enumerate}
\item Exam on Friday
\item Review in MEC from 12 to 1 today.
\item Office hours, 10am - 11am.
\end{enumerate}
\end{frame}

\section{Lecture}
\label{sec:org441add0}
\begin{frame}[label={sec:orgc9d7715}]{Modeling with Sinusoidal Functions}
Let's finish up that example from last class: Model the number of
hours of sunlight in a day \(t\) days after January 1.
\vspace{10in}
\end{frame}

\begin{frame}[label={sec:org1932edf}]{The basics of modeling with Sinusoidal Functions}
Anytime we are modeling a real world situation that is
\uline{\hspace*{1in}}, sinusoidal functions are what we use.  If we're trying to fit real world periodic data to a sinusoidal function, we need to know:

\begin{enumerate}
\item 

\item 

\item 

\item 
\end{enumerate}
\end{frame}

\begin{frame}[label={sec:org3087975}]{Example}
Let's model the average monthly temperature in Winchester, VA using
the information in the Google Sheets doc linked in Canvas.

\vspace{10in}
\end{frame}

\begin{frame}[label={sec:orgf529c23}]{Example}
\end{frame}
\end{document}