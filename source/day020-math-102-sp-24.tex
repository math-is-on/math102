% Created 2024-03-27 Wed 07:47
% Intended LaTeX compiler: pdflatex
\documentclass[presentation]{beamer}
\usepackage[utf8]{inputenc}
\usepackage[T1]{fontenc}
\usepackage{graphicx}
\usepackage{longtable}
\usepackage{wrapfig}
\usepackage{rotating}
\usepackage[normalem]{ulem}
\usepackage{amsmath}
\usepackage{amssymb}
\usepackage{capt-of}
\usepackage{hyperref}
\mode<beamer>{\usetheme{Madrid}}
\definecolor{SUred}{rgb}{0.59375, 0, 0.17969} % SU red (primary)
\definecolor{SUblue}{rgb}{0, 0.17578, 0.38281} % SU blue (secondary)
\setbeamercolor{palette primary}{bg=SUred,fg=white}
\setbeamercolor{palette secondary}{bg=SUblue,fg=white}
\setbeamercolor{palette tertiary}{bg=SUblue,fg=white}
\setbeamercolor{palette quaternary}{bg=SUblue,fg=white}
\setbeamercolor{structure}{fg=SUblue} % itemize, enumerate, etc
\setbeamercolor{section in toc}{fg=SUblue} % TOC sections
% Override palette coloring with secondary
\setbeamercolor{subsection in head/foot}{bg=SUblue,fg=white}
\setbeamercolor{date in head/foot}{bg=SUblue,fg=white}
\institute[SU]{Shenandoah University}
\titlegraphic{\includegraphics[width=0.5\textwidth]{\string~/Documents/suLogo/suLogo.pdf}}
\newcommand{\R}{\mathbb{R}}
\usetheme{default}
\author{Chase Mathison\thanks{cmathiso@su.edu}}
\date{27 March 2024}
\title{Solving Trigonometric Equations}
\hypersetup{
 pdfauthor={Chase Mathison},
 pdftitle={Solving Trigonometric Equations},
 pdfkeywords={},
 pdfsubject={},
 pdfcreator={Emacs 29.1 (Org mode 9.6.7)}, 
 pdflang={English}}
\begin{document}

\maketitle

\section{Announcements}
\label{sec:org28a8310}
\begin{frame}[label={sec:org5cb3918}]{Announcements}
\begin{enumerate}
\item Homework in M.O.M.
\item Office hours, 10am - 11am.
\end{enumerate}
\end{frame}

\section{Lecture}
\label{sec:org3d9b4d2}
\begin{frame}[label={sec:org93ee308}]{Solving Trig Equations}
Now we want to talk about how we might solve the equation
\[
\sin^2 (x) - 2\sin(x) + 1 = 0\]
We'll build up to this equation, but first let's start with some simple examples.
\vspace{10in}
\end{frame}

\begin{frame}[label={sec:org71ad25d}]{Example}
Solve the equation
\[
\sin(x) = \frac{1}{2}\]
for \(0 \le x < 2\pi\)

\vspace{10in}
\end{frame}

\begin{frame}[label={sec:org5dbb93b}]{Example}
\end{frame}

\begin{frame}[label={sec:org27aaf86}]{Example}
Solve the equation
\[
2\cos(x) - \sqrt{2} = 0 \]
for \(0 \le x < 2\pi\).
\vspace{10in}
\end{frame}

\begin{frame}[label={sec:orga8b973b}]{Example}
\end{frame}

\begin{frame}[label={sec:org592fb53}]{General Guidelines}
We'll use the following general guidelines when solving more complicated trigonometric
equations:

\begin{center}
\includegraphics[width=0.9\textwidth]{./pastClasses/solvingTrigEqs.png}
\end{center}
\end{frame}

\begin{frame}[label={sec:org9f2c26c}]{Example}
Solve the equation
\[
2\sin^2(x) - 1 =0\]
for \(0 \le x < 2\pi.\)

\vspace{10in}
\end{frame}

\begin{frame}[label={sec:org81b511b}]{Example}
\end{frame}

\begin{frame}[label={sec:orgd2f0bcd}]{Example}
Solve the equation
\[
\csc(x) = -2 \]
for \(0 \le x < 4\pi.\)
\vspace{10in}
\end{frame}

\begin{frame}[label={sec:org6f1f7a6}]{Example}
\end{frame}

\begin{frame}[label={sec:org1161b40}]{Example}
Sometimes we'll get solutions that don't involve special angles that we know,
That's okay!  We can just use a calculator and the inverse trig functions.

Solve:
\[
\sin(x) = 0.7\]
for \(0 \le x < 2\pi.\)

\vspace{10in}
\end{frame}

\begin{frame}[label={sec:org9a8e952}]{Example}
\end{frame}

\begin{frame}[label={sec:org30161f1}]{Example}
Time for a trickier one!

Solve
\[
\sin^2(x) - \sin(x) - 1 = 0\]
for \(0 \le x < 2\pi.\)


\vspace{10in}
\end{frame}

\begin{frame}[label={sec:orgc4f1bf5}]{Example}
\end{frame}

\begin{frame}[label={sec:orgffb62d5}]{Example}
Solve the following equation exactly:
\[
2\cos^2(x) + 3\cos(x) - 2 = 0\]
for \(0 \le x < 4\pi.\)
\vspace{10in}
\end{frame}

\begin{frame}[label={sec:orgb6b617b}]{Example}
\end{frame}

\begin{frame}[label={sec:orga3f3837}]{Example}
Solve
\[
2\sin(2x) + 1 = 0\]
for \(0 \le x < 2 \pi\).
\vspace{10in}
\end{frame}
\end{document}